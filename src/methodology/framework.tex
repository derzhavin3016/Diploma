\begin{tikzpicture} [
    scale=1,
    -Latex,auto,node distance =1 cm and .4 cm, thick,% node distance is the distance between one node to other, where 1.5cm is the length of the edge between the nodes
  ]
  % simulators

  \node[state, fill=Apricot] (s1) {Симулятор 1};
  \node[state, fill=Apricot] (s2) [below=.4cm of s1] {Симулятор 2};
  \node[state, fill=Apricot] (s3) [below=of s2] {Симулятор n};

  % interface
  \node[state] (int) [right=of s2, fill=LimeGreen] {Обобщённый \\интерфейс};

  \draw[->] (s1) -- (int);
  \draw[->] (s2) -- (int);
  \draw[->] (s3) -- (int);

  \path (s2) -- node[auto=false]{\ldots} (s3);


  \node[state] (dse) [right=of int, fill=LimeGreen] {Алгоритм \\подбора};
  \draw[->] (int) -- (dse);

  \node[state] (req) [above=of int, fill=cyan!50] {Запрос \\изменений};
  \draw[->] (dse) -- (req);
  \draw[->] (req) -- (int);

  \node[state] (change) [below=of int, fill=cyan!50] {Изменение \\параметров};

  \draw[->, thick, dashed, in=90] (s2.west) -- (s1.west);
  \draw[->, thick, dashed] (s3.west) -- (s2.west);
  \draw[->, thick, dashed] (change) -| (s3.west);

  \draw[->] (int) -- (change);

  \node[state] (traces) [below=of dse, fill=NavyBlue!60] {Трассы \\исполнения};
  \draw[->] (traces) -- (dse);

  \node[state] (stat) [above=of dse, fill=lime!50] {Статистика};
  \draw[->] (stat) -- (dse);

  % cluster
  \node[state, fill=NavyBlue!60] (t3) [right=.7cm of dse] {Трасса 3};
  \node[state, fill=NavyBlue!60] (t2) [above=.1cm of t3]  {Трасса 2};
  \node[state, fill=NavyBlue!60] (t4) [below=1.3cm of t3]  {Трасса M};
  \node[state, fill=NavyBlue!60] (t1) [above=.1cm of t2]  {Трасса 1};

  \path (t3) -- node[auto=false]{\ldots} (t4);

  \begin{scope}[on background layer]
    \node[draw, thick, fit=(t1)(t2)(t3)(t4), fill=Apricot] (cluster) [right=of dse] {};
  \end{scope}
  \draw[->, bend right] ([xshift=-1.1cm]cluster.north) to (stat.north);
  \node at (cluster.north) [above=.1cm of t1] {Кластер};

  \draw[->] (dse) -- (t3-|cluster.west);
\end{tikzpicture}
