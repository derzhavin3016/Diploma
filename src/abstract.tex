В работе предложен метод подбора оптимальной конфигурации микроархитектуры процессора, позволяющий варьировать большое число параметров малого порядка и эффективнее (по сравнению с существующими подходами) оценивать влияние этих параметров на производительность процессора. Для использования предложенного алгоритма реализована симуляторонезависмая среда моделирования с поддержкой параллельного запуска трасс исполнения. В результате тестовых запусков получено отклонение прироста производительности от максимально возможного (полученного методом полного перебора) на 3.2\%, при этом прирост производительности составил 2.22\%.
