В работе был предложен алгоритм оптимального подбора параметров, основанный на идее отображения статистических данных, полученных с симулятора, на параметры процессора.

Для использования и тестирования алгоритма была разработана среда моделирования с поддержкой обобщенного интерфейса работы с симуляторами. Данная среда моделирования также обладает возможностью запуска трасс исполнения в параллельном режиме.

С помощью тестовых запусков было выяснено, что предложенный алгоритм позволяет находить оптимальную конфигурацию микроархитектуры процессора через варьирование параметров низкого порядка.

В результате работы алгоритма на тестовом наборе приложений было получено среднее увеличение производительности на 2.22\%, при этом отклонение этого значения от максимально возможного прироста производительности, полученного методом полного перебора, составило 3.2\%.
