
В качестве симулятора производительности использовалась модель суперскалярного процессора с внеочередным исполнением команд архитектуры ARM~\cite{seal2001arm}, основанная на симуляторе gem5~\cite{binkert2011gem5}. В ходе работы в симулятор был добавлен подсчёт статистики для работы алгоритма подбора.

Запуски осуществлялись с помощью реализованной среды моделирования в режиме параллельного исполнения на наборе тестовых 3D\,--\,приложений.

Перед запуском была проведена предварительная обработка данных, с целью выделения групп (кластеров) приложений со схожими <<горячими>>\footnote{Значения, имеющие наибольшее относительное отклонение от своих пороговых значений} статистическими значениями. Для улучшения результатов кластеризации, была проведена нормализация данных. В результате предварительной обработки данных, было выделено два кластера, по которым в дальнейшем и производилось усреднение статистических значений.

Для оценки результата работы предложенного алгоритма, были проведены запуски в двух режимах:
\begin{itemize}
  \item Обычный режим (поиск оптимальной конфигурации с помощью предложенного алгоритма)
  \item Режим полного перебора~--- обход всех точек пространства конфигураций с нахождением максимально возможного IPC
\end{itemize}

В результате вышеописанных запусков, был получен прирост производительности, в среднем отклоняющийся на 3.2\% от оптимального прироста производительности (полученного методом полного перебора). Среднее улучшение производительности составило 2.22\%, причём для достижения этого результата потребовалось в среднем обойти всего 14\% от общего числа точек пространства конфигураций.

Полученный результат позволяет сделать вывод о том, что предложенный алгоритм способен успешно подбирать оптимальную конфигурацию процессора с производительностью, незначительно отличающейся от максимально возможной на заданном конфигурационном пространстве. Кроме того, алгоритму требуется значительно меньшее число точек конфигурационного пространства для обнаружения точки с оптимальной производительностью.
